% Copyright 2008 by Till Tantau
%
% This file may be distributed and/or modified
%
% 1. under the LaTeX Project Public License and/or
% 2. under the GNU Free Documentation License.
%
% See the file doc/generic/pgf/licenses/LICENSE for more details.


\section{Chains}

\label{section-chains}

\begin{tikzlibrary}{chains}
  This library defines options for creating chains.
\end{tikzlibrary}


\subsection{Overview}

\emph{Chains} are sequences of nodes that are -- typically -- arranged
in a row or a column and that are -- typically -- connected by
edges. More generally, they can be used to position nodes of a
branching network in a systematic manner. For the positioning of nodes
in rows and columns you can also use matrices, see
Section~\ref{section-matrices}, but chains can also be used
to describe the connections between nodes that have already been
connected using, say, matrices. Thus, it often makes sense to use
matrices for the positioning of elements and chains to describe the
connections.



\subsection{Starting and Continuing a Chain}

Typically, you construct one chain at a time, but it is
permissible to construct multiple chains simultaneously. In this
case, the chains must be named differently and you must specify for
each node which chain it belongs to.

The first step toward creating a chain is to use the |start chain|
option.

\begin{key}{/tikz/start chain=\opt{\meta{chain name}}\opt{\meta{direction}}}
  This key should, but need not, be given as an option to a scope
  enclosing all nodes of the chain. Typically, this will be a |scope|
  or the whole |tikzpicture|, but it might just be a path on which all
  nodes of the chain are found.
  If no \meta{chain name} is given, the default value |chain| will be
  used instead.

  The key starts a chain named \meta{chain name} and makes it
  \emph{active}, which means that it is currently being constructed. The
  |start chain| can be issued only once to activate a chain, inside a
  scope in which a chain is active you cannot use this option once
  more (for the same chain name). The chain stops being active at the
  end of the scope in which the |start chain| command was given.

  Although chains are only locally active (that is, active inside the
  scope the |start chain| command was issued), the information
  concerning the chains is stored globally and it is possible
  to \emph{continue} a chain after a scope has ended. For this, the
  |continue chain| option can be used, which allows you to reactivate
  an existing chain in another scope.

  The \meta{direction} is used to determine the placement rule for
  nodes on the chain. If it is omitted, the current value of the
  following key is used:
  \begin{key}{/tikz/chain default direction=\meta{direction}
      (initially going right)}
    This \meta{direction} is used in a |chain| option, if no other
    \meta{direction} is specified.
  \end{key}

  The \meta{direction} can have two different forms:
  \declare{|going |\meta{options}} or
  \declare{|placed |\meta{options}}. The effect of these rules will be
  explained in the description of the |on chain| option. Right now,
  just remember that the \meta{direction} you provide with the |chain|
  option applies to the whole chain.

  Other than this, this key has no further effect. In particular, to
  place nodes on the chain, you must use the |on chain| option,
  described next.
\begin{codeexample}[]
\begin{tikzpicture}[start chain]
  % The chain is called just "chain"
  \node [on chain] {A};
  \node [on chain] {B};
  \node [on chain] {C};
\end{tikzpicture}
\end{codeexample}

\begin{codeexample}[]
\begin{tikzpicture}
  % Same as above, using the scope shorthand
  { [start chain]
    \node [on chain] {A};
    \node [on chain] {B};
    \node [on chain] {C};
  }
\end{tikzpicture}
\end{codeexample}

\begin{codeexample}[]
\begin{tikzpicture}[start chain=1 going right,
                    start chain=2 going below,
                    node distance=5mm,
                    every node/.style=draw]
  \node [on chain=1] {A};
  \node [on chain=1] {B};
  \node [on chain=1] {C};

  \node [on chain=2] at (0.5,-.5) {0};
  \node [on chain=2] {1};
  \node [on chain=2] {2};

  \node [on chain=1] {D};
\end{tikzpicture}
\end{codeexample}
\end{key}

\begin{key}{/tikz/continue chain=\opt{\meta{chain name}}\opt{\meta{direction}}}
  This option allows you to (re)activate an existing chain and to
  possibly change the default direction. If the |chain name| is
  missing, the name of the innermost activated chain is used. If no
  chain is activated, |chain| is used.

  Let us have a look at the two different applications of this
  option. The first is to change the direction of a chain as it is
  being constructed. For this, just give this option somewhere inside
  the scope of the chain.
\begin{codeexample}[]
\begin{tikzpicture}[start chain=going right,node distance=5mm]
  \node [draw,on chain] {Hello};
  \node [draw,on chain] {World};
  \node [draw,continue chain=going below,on chain] {,};
  \node [draw,on chain] {this};
  \node [draw,on chain] {is};
\end{tikzpicture}
\end{codeexample}

  The second application is to reactivate a chain after it ``has
  already been closed down.''

\begin{codeexample}[]
\begin{tikzpicture}[node distance=5mm,
                    every node/.style=draw]
  { [start chain=1]
    \node [on chain] {A};
    \node [on chain] {B};
    \node [on chain] {C};
  }

  { [start chain=2 going below]
    \node [on chain=2] at (0.5,-.5) {0};
    \node [on chain=2] {1};
    \node [on chain=2] {2};
  }

  { [continue chain=1]
    \node [on chain] {D};
  }
\end{tikzpicture}
\end{codeexample}
\end{key}


\subsection{Nodes on a Chain}

\begin{key}{/tikz/on chain=\opt{\meta{chain name}}\opt{\meta{direction}}}
  This key should be given as an option to a node. When the option is
  used, the \meta{chain name} must be the name of a chain that has
  been started using the |start chain| option. If \meta{chain name} is
  the empty string, the current value of the innermost activated chain
  is used. If this option is used several times for a node, only the
  last invocation ``wins.'' (To place a node on several chains, use
  the |\chainin| command repeatedly.)

  The \meta{direction} part is optional. If present, it sets the
  direction used for this node, otherwise the \meta{direction}
  that was given to the original |start chain| option is used (or of
  the last |continue chain| option, which allows you to change this
  default).

  The effects of this option are the following:
  \begin{enumerate}
  \item An internal counter (there is one local counter
    for each chain) is increased. This counter reflects the current
    number of the node in the chain, where the first node is node 1,
    the second is node 2, and so on.

    The value of this internal counter is globally stored in the
    macro \declare{|\tikzchaincount|}.
  \item If the node does not yet have a name, (having been given using
    the |name| option or the name-syntax), the name of the node is set to
    \meta{chain name}|-|\meta{value of the internal chain
      counter}. For instance, if the chain is called |nums|, the first
    node would be named |nums-1|, the second |nums-2|, and so on. For
    the default chain name |chain|, the first node is named |chain-1|,
    the second |chain-2|, and so on.
  \item Independently of whether the name has been provided
    automatically or via the |name| option, the name of the node is
    globally stored in the macro \declare{|\tikzchaincurrent|}.
  \item Except for the first node, the macro
    \declare{|\tikzchainprevious|} is now globally set to the name of
    the node of the previous node on the chain. For the first node of
    the chain, this macro is globally set to the empty string.
  \item Except possibly for the first node of the chain, the placement
    rule is now executed. The placement rule is just a \tikzname\ option
    that is applied automatically to each node on the chain. Depending
    on the form of the \meta{direction} parameter (either the locally
    given one or the one given to the |start chain| option), different
    things happen.

    First, it makes a difference whether the \meta{direction} starts
    with |going| or with |placed|. The difference is that in the first
    case, the placement rule is not applied to the first node of the
    chain, while in the second case the placement rule is applied also
    to this first node. The idea is that a |going|-direction indicates
    that we are ``going somewhere relative to the previous node''
    whereas a |placed| indicates that we are ``placing nodes according
    to their number.''

    Independently of which form is used, the \meta{text} inside
    \meta{direction} that follows |going| or |placed| (separated by a
    compulsory space) can have two different effects:
    \begin{enumerate}
    \item If it contains an equal sign, then this \meta{text} is used
      as the placement rule, that is, it is simply executed.
    \item If it does not contain an equal sign, then
      \meta{text}|=of \tikzchainprevious| is used as the placement
      rule.
    \end{enumerate}

    Note that in the first case, inside the \meta{text} you have
    access to |\tikzchainprevious| and |\tikzchaincount| for doing
    your positioning calculations.
\begin{codeexample}[]
\begin{tikzpicture}[start chain=circle placed {at=(\tikzchaincount*30:1.5)}]
  \foreach \i in {1,...,10}
    \node [on chain] {\i};

  \draw (circle-1) -- (circle-10);
\end{tikzpicture}
\end{codeexample}
  \item
    The following style is executed:
    \begin{stylekey}{/tikz/every on chain}
      This key is executed for every node on a chain, including the
      first one.
    \end{stylekey}
  \end{enumerate}

  Recall that the standard placement rule has a form like
  |right=of (\tikzchainprevious)|. This means that each
  new node is placed to the right of the previous one, spaced by the
  current value of |node distance|.
\begin{codeexample}[]
\begin{tikzpicture}[start chain,node distance=5mm]
  \node [draw,on chain] {};
  \node [draw,on chain] {Hallo};
  \node [draw,on chain] {Welt};
\end{tikzpicture}
\end{codeexample}

  The optional \meta{direction} allows us to temporarily change the
  direction in the middle of a chain:
\begin{codeexample}[]
\begin{tikzpicture}[start chain,node distance=5mm]
  \node [draw,on chain] {Hello};
  \node [draw,on chain] {World};
  \node [draw,on chain=going below] {,};
  \node [draw,on chain] {this};
  \node [draw,on chain] {is};
\end{tikzpicture}
\end{codeexample}

  You can also use more complicated computations in the \meta{direction}:
\begin{codeexample}[]
\begin{tikzpicture}[start chain=going {at=(\tikzchainprevious),shift=(30:1)}]
  \draw [help lines] (0,0) grid (3,2);
  \node [draw,on chain] {1};
  \node [draw,on chain] {Hello};
  \node [draw,on chain] {World};
  \node [draw,on chain] {.};
\end{tikzpicture}
\end{codeexample}
\end{key}

For each chain, two special ``pseudo nodes'' are created.

\begin{predefinednode}{\meta{chain name}-begin}
  This node is the same as the first node on the chain. It is only
  defined after a first node has been defined.
\end{predefinednode}

\begin{predefinednode}{\meta{chain name}-end}
  This node is the same as the (currently) last node on the chain. As
  the chain is extended, this node changes.
\end{predefinednode}

The |on chain| option can also be used, in conjunction with
|late options|, to add an already existing node to a chain. The
following command, which is only defined inside scopes where a
|start chain| option is present, simplifies this process.

\begin{command}{\chainin |(|\meta{existing name}|)| \opt{\oarg{options}}}
  This command makes it easy to add a node to chain that has already
  been constructed. This node may even be part of a another chain.

  When you say |\chainin (some node);|, the node |some node| must
  already exist. It will then be made part of the current chain. This
  does not mean that the node can be changed (it is already
  constructed, after all), but the |join| option can be used to join
  |some node| to the previous last node on the chain and subsequent
  nodes will be placed relative to |some node|.

  It is permissible to give the |on chain| option inside the
  \meta{options} in order to specify on which chain the node should be
  put.

  This command is just a shortcut for
\begin{quote}
|\path (|\meta{existing name}|) [late options={on chain,every chain in,|\meta{options}|}]|
\end{quote}
  In particular, it is possible to continue to path after a |\chainin|
  command, though that does not seem very useful.

\begin{codeexample}[]
\begin{tikzpicture}[node distance=5mm,
                    every node/.style=draw,every join/.style=->]
  \draw [help lines] (0,0) grid (3,2);

  \node[red] (existing) at (0,2) {existing};

  { [start chain]
    \node [draw,on chain,join] {Hello};
    \node [draw,on chain,join] {World};
    \chainin (existing) [join];
    \node [draw,on chain,join] {this};
    \node [draw,on chain,join] {is};
  }
\end{tikzpicture}
\end{codeexample}

  Here is an example where nodes are positioned using a matrix and
  then connected using a chain
{\catcode`\|=12
\begin{codeexample}[]
\begin{tikzpicture}[every node/.style=draw]
  \matrix [matrix of nodes,column sep=5mm,row sep=5mm]
  {
    |(a)|           World & |(b) [circle]|             peace \\
    |(c)|           be    & |(d) [isosceles triangle]| would \\
    |(e) [ellipse]| great & |(f)|                      ! \\
  };

  { [start chain,every on chain/.style={join=by ->}]
    \chainin (a);
    \chainin (b);
    \chainin (d);
    \chainin (c);
    \chainin (e);
    \chainin (f);
  }
\end{tikzpicture}
\end{codeexample}
}
\end{command}



\subsection{Joining Nodes on a Chain}

\begin{key}{/tikz/join=\opt{|with |\meta{with} }\opt{|by |\meta{options}}}
  When this key is given to any node on a chain (except possibly for
  the first node), an |edge| command is added after the node. The
  |with| part specifies which node should be used for the start point
  of the edge; if the |with| part is omitted, the |\tikzchainprevious|
  is used. This |edge| command gets the \meta{options} as parameter
  and the current node as its target. If there is no
  previous node and no |with| is given, no |edge| command gets
  executed.
  \begin{stylekey}{/tikz/every join}
    This style is executed each time this command is used.
  \end{stylekey}

  Note that it makes sense to call this option several times for a
  node, in order to connect it to several nodes. This is especially
  useful for joining in branches, see the next section.
\begin{codeexample}[]
\begin{tikzpicture}[start chain,node distance=5mm,
                    every join/.style={->,red}]
  \node [draw,on chain,join] {};
  \node [draw,on chain,join] {Hallo};
  \node [draw,on chain,join] {Welt};
  \node [draw,on chain=going below,
         join,join=with chain-1 by {blue,<-}] {foo};
\end{tikzpicture}
\end{codeexample}
\end{key}


\subsection{Branches}

A \emph{branch} is a chain that (typically only temporarily) extends
an existing chain. The idea is the following: Suppose we are
constructing a chain and at some node |x| there is a fork. In this
case, one (or even more) branches starts at this fork. For each branch
a chain is created, but the first node on this chain should
be~|x|. For this, it is useful to use |\chainin| on the node |x| to
make it part of the different branch chains and to name the branch
chains in some way that reflects the name of the main chain.

The |start branch| option provides a shorthand for doing exactly what
was just described.

\begin{key}{/tikz/start branch=\meta{branch name}\opt{\meta{direction}}}
  This key is used in the same manner as the |start chain| command,
  however, the effect is slightly different:
  \begin{itemize}
  \item This option may only be used if some chain is already active
    and there is a (last) node on this chain. Let us call this node
    the \meta{fork node}.
  \item The chain is not just called \meta{branch name}, but
    \meta{current chain}|/|\meta{branch name}. For instance, if the
    \meta{fork node} is part of the chain called |trunk| and the
    \meta{branch name} is set to |left|, the complete chain name of
    the branch is |trunk/left|. The \meta{branch name} must be given,
    there is no default value.
  \item The \meta{fork node} is automatically ``chained into'' the
    branch chain as its first node. Thus, for the first node on the
    branch that you provide, the |join| option will cause it to be
    connected to the fork node.
  \end{itemize}
\begin{codeexample}[]
\begin{tikzpicture}[every on chain/.style=join,every join/.style=->,
                    node distance=2mm and 1cm]
  { [start chain=trunk]
    \node [on chain] {A};
    \node [on chain] {B};

    { [start branch=numbers going below]
      \node [on chain] {1};
      \node [on chain] {2};
      \node [on chain] {3};
    }
    { [start branch=greek going above]
      \node [on chain] {$\alpha$};
      \node [on chain] {$\beta$};
      \node [on chain] {$\gamma$};
    }

    \node [on chain,join=with trunk/numbers-end,join=with trunk/greek-end] {C};
    { [start branch=symbols going below]
      \node [on chain] {$\star$};
      \node [on chain] {$\circ$};
      \node [on chain] {$\int$};
    }
  }
\end{tikzpicture}
\end{codeexample}
\end{key}

\begin{key}{/tikz/continue branch=\meta{branch name}\opt{\meta{direction}}}
  This option works like the |continue chain| option, only
  \meta{current chain}|/|\meta{branch name} is used as the chain name,
  rather than just \meta{branch name}.
\begin{codeexample}[]
\begin{tikzpicture}[every on chain/.style=join,every join/.style=->,
                    node distance=2mm and 1cm]
  { [start chain=trunk]
    \node [on chain] {A};
    \node [on chain] {B};
    { [start branch=numbers going below] } % just a declaration,
    { [start branch=greek   going above] } % we will come back later
    \node [on chain] {C};

    % Now come the branches...
    { [continue branch=numbers]
      \node [on chain] {1};
      \node [on chain] {2};
    }
    { [continue branch=greek]
      \node [on chain] {$\alpha$};
      \node [on chain] {$\beta$};
    }
  }
\end{tikzpicture}
\end{codeexample}
\end{key}
